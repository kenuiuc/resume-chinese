\documentclass[alternative]{resume_template}
\name{王科}{}
\socialinfo{
\smartphone{130-7355-0335}\email{kenuiuc@gmail.com}\infos{微信:ken\_uiuc}
\address{深圳}\linkedin{kenuiuc}\github{kenuiuc}
}
\photo{2.5cm}{ken_linkedin_avatar.jpeg}
\begin{document}
\makecvheader

% Summary
\begin{itemize}
    \item 本科毕业于美国计算机专业Top5学校,硕士毕业于加拿大top2学校
    \item 3年 Java 后端开发经验,熟练掌握基于 Spring Boot 和 Spring Cloud 的微服务开发
    \item 长期活跃于 Java 开发者论坛 Together Java, 为初学者解答大量问题
    \item 具备基本的机器学习和数据分析技能,会使用 Python, R 和 AWS SageMaker
\end{itemize}

% Experience
\sectionTitle{工作}{\faSuitcase}
\begin{experiences}

    \experience
    {2022 - 2024}{加拿大}
    {专业进修}{英属哥伦比亚大学 | UBC}
    {
        \begin{itemize}
            \item 数据科学硕士研究生
            \item 计算机课程助教,软件开发培训
        \end{itemize}
    }
    {机器学习, 深度学习, 大语言模型, 数据分析, Python, pandas, scikit-learn, PyTorch, AWS SageMaker}

    \experience
    {2021 - 2021}{北京}
    {Java 软件工程师}{环信/声网 Agora Inc}
    {
        \begin{itemize}
            \item 开发基于微服务架构和 Spring Cloud 的 SaaS 即时消息系统
            \item 使用字典树和 Thrift RPC 实现了一个高性能关键词检测程序,用来过滤违规消息
            \item 利用 Reactor Netty 客户端开发了一套响应式的 Java SDK,方便用户对消息平台进行管理操作
            \item 为满足 GDPR 合规要求,开发了一个信封加密库对用户数据加密处理,并使用 AWS KMS 管理密钥
        \end{itemize}
    }
    {Spring Boot, Spring Cloud, MySQL, Redis, Tomcat, Netty, Reactor, GitHub Actions}

    \experience
    {2020 - 2021}{北京}
    {Java 软件工程师}{数语科技}
    {
        \begin{itemize}
            \item 开发类似 ERwin Data Modeler 的数据治理工具,用于对大规模数据库的元数据管理
            \item 完善了数据质量检测功能——系统可扫描数据库,并使用可定制的规则验证数据的合理性
            \item 为实现与第三方系统的数据对接,使用 Spring MVC 开发了一套 REST API
            \item 使用 JUnit 和 Mockito 为现有功能完善单元测试,使覆盖率达到80\%以上
        \end{itemize}
    }
    {Spring MVC, JDBC, PostgreSQL, Oracle, MySQL, Hibernate, JUnit, Mockito, Maven}

    \experience
    {2019 - 2020}{北京}
    {全栈程序员}{ShopHitly}
    {
        \begin{itemize}
            \item 参与早期的电商平台创业项目,向美国伊利诺伊州用户销售烟具
            \item 使用 Node.js, Express.js 和 MongoDB 搭建 REST web 服务,用来管理商品库存信息
            \item 用 Docker, AWS ECS 和 CodePipeline 搭建 CI/CD 管道并部署应用
            \item 基于 TypeScript 和 React 搭建简单的 web UI
        \end{itemize}
    }
    {Node.js, Express.js, MongoDB, TypeScript, React, HTML, CSS, Docker, AWS ECS, REST}

    \experience
    {2018 - 2019}{美国}
    {Hadoop 数据工程师}{雅虎/Verizon (全美最大的电信公司)}
    {
        \begin{itemize}
            \item 负责维护一个低延迟 OLAP 数据仓库系统,用于网站的用户参与分析 (user engagement)
            \item 使用 Hadoop 大数据技术栈搭建 ETL 数据管道
            \item 将数据导入高性能的 Apache Druid, 使用户可在几秒内对PB量级的数据进行聚合查询
            \item 开发 Java web 服务,为查询 Druid 数据提供 REST API
        \end{itemize}
    }
    {Hadoop, Java, Apache Druid, Apache Pig, Oozie, Hive, SLF4J, H2, Gradle, Linux}

\end{experiences}

%Section: Education and Internships
\twocolumnsection
{\sectionTitle{学历}{\faMortarBoard}
    \begin{internships}
        \internship
        {2023}{数据科学,硕士}
        {英属哥伦比亚大学 | UBC(世界排名第38)}
        \vspace{5pt}
        \internship
        {2017}{计算机科学,本科}
        {伊利诺伊大学香槟分校 | UIUC(美国专业排名第5)}
    \end{internships}
}
{\sectionTitle{实习}{\faMortarBoard}
    \begin{internships}
        \internship
        {2017}{Java 程序员}
        {谷歌/CDAP 数据平台}
        \vspace{5pt}
        \internship
        {2016}{Java 程序员}
        {芝商所/CME Group (全球最大期货交易所)}
    \end{internships}
}

\end{document}
